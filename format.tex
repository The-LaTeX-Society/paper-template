% ----- Standard IEEEtran Formatting -----

%\IEEEoverridecommandlockouts
% The preceding line is only needed to identify funding in the first footnote. If that is unneeded, please comment it out.
\usepackage{cite}
\usepackage{amsmath,amssymb,amsfonts}
\usepackage{algorithmic}
\usepackage{graphicx}
\usepackage{textcomp}
\usepackage[table]{xcolor}
\def\BibTeX{{\rm B\kern-.05em{\sc i\kern-.025em b}\kern-.08em
		T\kern-.1667em\lower.7ex\hbox{E}\kern-.125emX}}

% ----- Custom Formatting -----

% --- Language (can be unneccesary when using babel-package) ---

% Rename "Abstract"
%\renewcommand{\abstractname}{Zusammenfassung}

% Rename "Index Terms" from IEEEkeywords
\renewcommand{\IEEEkeywordsname}{Schlüsselworte}

% Rename "References"
%\renewcommand{\refname}{Literatur}

% Rename "Fig."
%\renewcommand{\figurename}{Abbildung}

% --- Language and Hyphenation ---

% Correct german hyphenation and renaming of certain keywords
\usepackage[ngerman]{babel}

% Rename "Zusammenfassung" in babel-package
\addto{\captionsngerman}{\renewcommand{\abstractname}{Abstract}}

% --- References ---

\usepackage[hidelinks]{hyperref}

% --- Drawing ---

\usepackage{tikz}

% --- Bibliography ---

\bibliographystyle{IEEEtranDE}

% --- Listings ---

\usepackage{listings}
\lstset{
	frame=single,
	language=C++,
	basicstyle=\ttfamily\footnotesize,
	showstringspaces=false,
	tabsize=4
}

\makeatletter
\def\lst@makecaption{%
	\def\@captype{table}%
	\@makecaption
}
\makeatother

% --- PDF Metadata ---

\hypersetup{
	pdfauthor={Autor},
	pdftitle={Titel},
	pdfsubject={Thema},
	pdfkeywords={Stichwörter},
	bookmarksnumbered
}

% --- Tables ---

\usepackage{booktabs}
\usepackage{multirow}
\usepackage{makecell}

% --- Prevent widows and orphans ---

\usepackage[all]{nowidow}